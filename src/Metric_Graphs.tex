\documentclass[a4paper,10pt]{book}

\usepackage{textcomp}
\usepackage{amsmath}
\usepackage{amsthm}
\usepackage[top=0.5in, bottom=0.5in, left=0.5in, right=0.5in]{geometry}

\newtheorem{theorem}{Theorem}[chapter]
\newtheorem{lemma}[theorem]{Lemma}

\theoremstyle{definition}
\newtheorem{definition}[theorem]{Definition}
\newtheorem{example}[theorem]{Example}
\newtheorem{xca}[theorem]{Exercise}
\newtheorem{con}[theorem]{}

\theoremstyle{remark}
\newtheorem{remark}[theorem]{Remark}

\numberwithin{section}{chapter}
\numberwithin{equation}{chapter}

\begin{document}
 
 \chapter{Metric Graphs}
 \section{Graphs: Combinatorical Structure}
 
 \begin{definition}
  A graph \emph{$\Gamma$} consists of
  
  \begin{itemize}
   \item A pair of sets $(V,\ E)$ (called the `Vertex' and `Edge' sets respectively)
   \item An involution  $\: \phi \colon E \rightarrow E$
               satisfying $\phi (e) = \bar{e} $ and $\phi\circ\phi = \mathbf{1}_E$ (identity function)
   \item Functions $i,\ \tau: E \rightarrow V$ such that $i(\bar{e}) = \tau (e) $ and $\tau(\bar{e}) = i(e)$
  \end{itemize}

 \end{definition}

 Insert EXAMPLE 'ERE
 
 \section{A Graph as a Space}

   The following construction associates to a graph $\Gamma$, a space $| \Gamma |$ 
   \vspace{0.1in} \\ Let $ \xi = E \times [0,1] = \{ (e,t) : e \in E,\ t \in [0,1]\} $.\\
   Let $ \widetilde{\Gamma} =  \xi \cup V $ \\
   Let `\texttildelow' be the equivalence relation on $ \widetilde{\Gamma} $ generated by: 
   \begin{itemize}
      \item If $ (e,t) \in \xi $, then $ (e,t) \sim (\bar{e},1-t) $
      \item $ \forall \ e \in E $, $ i(e) \sim (e,0) $ and $ \tau (e) \sim (e,1) $
   \end{itemize}
   $ |\Gamma| $ is the quotient space $ \Gamma / \sim $
   \section{ Metric structure on a Graph}
   
   \subsection{Distance on the Vertex set V}
   \begin{definition}
      Two vertices $ v_{1}$ and $v_{2} $ are said to be \emph{adjacent} if there exists $ e \in E $ such that $ i(e) = v_1 $ and $ \tau (e) = v_2 $.
   \end{definition}
   Let $ \cal{D} = \{ $ d $: V \times V \rightarrow \mathbf{R} \ : \ $ 
 d is a metric; d($ v_1,\ v_2 $) $ \leq $ 1 if $ v_1 $, $ v_2 $ are adjacent $ \} $
   \begin{lemma}
      $ \cal{D} $ is not empty
   \end{lemma}
   \begin{definition}
      The \emph{Graph Metric} on V is defined as
      \begin{equation}
	  d_{max}(v_1,v_2)= \sup_{d \in \cal{D} }\{d(v_1,v_2) \} \nonumber
      \end{equation} 
   \end{definition}
   \begin{lemma}
	$ d_{max} $ is a metric.
    \proof 
    \begin{enumerate}
      \item $\forall x \in V,\ d_{max}(x, x)= \sup_{d \in \cal{D}}\{d(x,x) \} = \sup \{ 0 \} = 0 $
      \item $\forall x,\ y \in V,\ d_{max}(x, y)= \sup_{d \in \cal{D} }\{d(x,y) \} = \sup_{d \in \cal{D} }\{d(y,x) \} =d _{max}(y, x) $
      \item $d(x, z)$
    \end{enumerate}    
  \end{lemma}
\end{document}
